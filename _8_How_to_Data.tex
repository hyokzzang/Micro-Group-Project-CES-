\section{How to take it to Data?}
\subsection{Nested CES Production Functions}
Recent immigration literature [see \cite{Borjas2014} and \cite{LewisPeri2015}] has emphasized the use of a special case of the CES Production function called the nested CES production function.  This nested function can be expressed as follows:
\begin{equation}
Q_t = [\lambda_{Kt} K_t^\delta + \lambda_{Lt}L_t^\delta]^\frac{1}{\delta}
\end{equation}
where:
\begin{equation}
L_t = [\sum \theta_{et} L_{et}^\beta]^\frac{1}{B}
\end{equation}
and
\begin{equation}
L_{et} = [\sum \alpha_{ex} L_{ext}^\gamma]^\frac{1}{\gamma}
\end{equation}
$L_t$ is the aggregate supply of workers, $L_{et}$ is the supply of workers with education e at time t, and $L_{ext} $ is the number of workers in education group e, experience group x, at time t.  Using this framework, each labor input can be expressed as a function of other labor inputs (i.e. the nesting structure). $L_{ext}$ can be further broken down to represent workers in each education/experience group who are either immigrant or native workers at time t.  This framework can be used to derive demand functions for each group of worker where elasticities of substitution for each subgroup of workers can be estimated empirically. One limit to this approach is that is imposes constant elasticities of substitution between groups in each nest.  For example, the elasticity of substitution between high school dropouts and people with graduate degrees is the same as the elasticity of substitution between high school dropouts and high school graduates.  This restriction may be less than ideal, yet it allows for the estimation of the elasticities with samples of data which is desirable.\par
This literature uses a variety of nesting structures in an attempt to identify the effect of immigration on native labor market outcomes (particularly wages and employment levels).

\begin{flushleft}
The type of nesting structure will depend on the question being asked by the researcher and will vary depending on the researcher's understanding of economic theory.  Since the aggregate labor input is itself a function of other labor inputs, the way these nests are determined will have a direct impact on the interpretation of the parameters that are estimated empirically.  For instance, it may be more reasonable to impose a constant elasticity of substitution between individuals with a college education and those without a college education since we may believe that it is generally constant (or falls within a small range).  However, it may be unrealistic to believe that this elasticity of substitution is the same between between high-school dropouts and those with graduate degrees.  So the researcher must be careful when determining the level of depth in his nesting structure.  Nevertheless, there may be circumstance when it may be reasonable to impose this restriction on other groups of workers.  For example, imposing this constant elasticity of substitution between workers with 15 and 20 years of experience and those with 15 and 25 years of experience may not be too unrealistic.
\end{flushleft}

\subsection{Jobs for Later}
\cite{ACMS1961}, DSGE papers, and some paper from empirical IO.
