\section{Review of Important Literature regarding CES}
\cite{Uzawa1962} was one of the firsts to develop extended theoretical versions of the CES. The author extended the previous production function to n input factors and found that the partial elasticities of substitution are independent of factors prices and are identical for all pairs of two factors of production. The use of the CES functional form for more than 2 factors will generally mean that there is not constant elasticity of substitution among all factors. This was a this was one of the earliest and most proeminent debates in CES literature with other attempts of generalization being made by \cite{mcfadden1963constant}, \cite{mukerji1963generalized}
and finally \cite{sato1967}. \par

Another important contribution on the theoretical side comes from \cite{revankar1971class} who develops a more general case to of CES, the  Variable Elasticity of Substiution function. The so called VES is nothing more than a production function for which the substitution parameter varies linearly with the capital-larbor ratio. Hence the VES poses a linear view of the world, opposed to the log-linear view of the CES. \par

On the practical side the CES was fundamental to the construction of the monopolistic competition model seen in \cite{DixitStiglitz1977}. Being one of the first to extrapolate the CES from a mere production function, authors develop models to study various aspects of the relationship between market and optimal resource allocation. They argue that monopoly power and the direction of market distortion is no longer clear. In fact, in the case of variable elasticity of substitution the bias could go either way but on the case of constant elasticity the  the market solution will be a constrained Pareto optimal.\par

Apart from monopolist competition, the CES function has also been applied to several other areas. \cite{miles1978} currency substitution model. On that model level of financial services are assumed to follow a Constant Elasticity Function. In a world where individuals also hold foreign currencies diversified portfolios will imply that monetary policy will produce changes in the interest rate that induce offsetting money flows even under a flexible exchange rates regime. \cite{Krugman1991}. The CES was also used in a similar context by \cite{Armington1969} \cite{Krugman1991}, \cite{Melitz2003} among others.

On a general way, several micro foundated macro models rely on CES functions:  \cite{Kydland1982} and a large part of the DSGE literature that followed it also made use these functions. \cite{BlanchardKiyotaki1987} microfunded macro model also assumes monopolistic competition and uses CES production and demand functions for consumption goods. In other words the popularization of these sort of models allowed CES functions to be applied on all areas not only as a production function but also as modeling consumer's preferences. 