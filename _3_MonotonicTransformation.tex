%%%%%%%%%%%%%%%%%%%%%%%%%%%%% INPUT FILE %%%%%%%%%%%%%%%%%%%%%%%%%%%%%

\section{Typical Monotone Increasing Transformation}
A typical monotonic increasing transformation for CES functions is applying the natural logarithm to the CES function. Any monotonic transformation of a homothetic function is homothetic. Any monotonic increasing transformation of the CES function would exhibit the same isoquants, and so, the same elasticity of substitution.
	
In the case of CES preference ordering (i.e. demand? utility?), monotonic transformations of the production function do not matter at all.
	
In addition to taking the natural logarithm of the CES function, another monotonic increasing transformation is taking the entire function to the power of $\beta$.{ This results in the "general" form of the utility function as seen in Part 1.}

\subsection{Log version}
Log-CES function, which will be extensively used in Section \ref{sec:special_case}, can be one example of monotone increasing transformation of the CES function.
\begin{IEEEeqnarray}{rCl}
\ln f(x) & = & \frac{1}{\sigma} \ln \left( \sum_{l=1}^L a_l x_l^{\sigma} \right) \label{eq:CES_monotone1}
\end{IEEEeqnarray}
First thing to check is that, does log-transformation change the preference order of variables? To see this we should check if $x \prec x'$ implies $\ln(f(x)) \leq \ln(f(x'))$. This is trivial since log is monotone increasing transformation, so $f(x)) \leq f(x')$ implies $\ln(f(x)) \leq \ln(f(x'))$. The log transformation also satisfies the homotheticity.

\subsection{Scaling of Parameters}
In this section we apply a monotonic increasing transformation $g(\cdot)$ which is defined as $g(x) = x^{\beta}$ where $\beta>0$. Then, applying this monotonic transformation, the CES function can be rewritten as,
\begin{IEEEeqnarray}{rCl}
    \hat{f}(x) = \left( \sum_{l=1}^L a_l x_l^{\sigma} \right)^{\frac{\beta}{\sigma}} \label{eq:CES_monotone2}
\end{IEEEeqnarray}
One big difference made by introducing $\beta$ is that the $\hat{f}(x)$ becomes homogeneous to degree $\beta$ with respect the $x$, that is $\beta\hat{f}{x} = \hat{f}{\beta x}$ since
\begin{IEEEeqnarray}{rCl}
    \hat{f}(\beta x) & = & \left( \sum_{l=1}^L a_l (\beta x_l)^{\sigma} \right)^{\frac{\beta}{\sigma}} \nonumber \\
    & = & \beta^{\frac{\sigma \beta}{\sigma}} \left( \sum_{l=1}^L a_l x_l^{\sigma} \right)^{\frac{\beta}{\sigma}} \nonumber \\
    & = & \beta \left( \sum_{l=1}^L a_l x_l^{\sigma} \right)^{\frac{\beta}{\sigma}} \nonumber \\ 
    & = & \beta \hat{f}(x) \label{eq:CES_monotone2}
\end{IEEEeqnarray}

What happens to the marginal rate of substitution, $\text{MRS}_{jk}$, between $x_j$ and $x_k$? Since,
\begin{IEEEeqnarray}{rCl}
    \frac{\partial \hat{f}(x) / \partial x_j}{\partial \hat{f}(x) / \partial x_k} & = & \frac{\beta \left(\sum_{l=1}^{L}a_l x_l^{\sigma} \right)^{\frac{\beta-\sigma}{\sigma}} a_j x_j^{\sigma-1}}{\beta \left(\sum_{l=1}^{L}a_l x_l^{\sigma} \right)^{\frac{\beta-\sigma}{\sigma}} a_k x_k^{\sigma-1}} \nonumber \\
    & = & \frac{a_j x_j^{\sigma-1}}{a_k x_k^{\sigma-1}} \nonumber \\
    & = & \frac{\partial f(x)/ \partial x_j}{\partial f(x)/ \partial x_k} \nonumber
\end{IEEEeqnarray}
we can see that the marginal rate of substitution is the same before taking the increasing monotone transformation. Since for all $j$, and $k$, this ordinal property of the CES utility function holds, this monotonic transformation do not change the optimal bundle of consumption in the utility maximization problem.

However, since the utility function is no longer homogeneous of degree one, this is cardinal property, and the value function will also be a homogeneous of degree $\beta$.

What happens when $\beta = \sigma$? Then the utility or production function becomes linearly separable with respect to $x$ which makes cross effects zero, $\partial^2 \hat{f}(x) / \partial x_l \partial x_k =0 $. In this case, utility ordering will not depend on the level of individual commodities, that is for $x = (\alpha, x_2, \cdots, x_L)$, and $y = (\alpha, y_2, \cdots, y_L)$, and $x \succ y$, will also imply $x' \succ y'$ where $x' = (\beta, x_2, \cdots, x_L)$, $y' = (\beta, y_2, \cdots, y_L)$, and $\alpha \neq \beta$.
%%%%%%%%%%%%%%%%%%%%%%%%%%%%%%%%%%%%%%%%%%%%%%%%%%%%%%%%%%%%%%%%%%%%% 