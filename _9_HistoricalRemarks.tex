%%%%%%%%%%%%%%%%%%%%%%%%%%%%%%%%%%%%%%%%%%%%%%%%%%%%%%%%%%%%%%%%%%%%%%%
\section{Historical Remarks}
Historically, \cite{Allen1938} was the first to derive the elasticity of demand for a factor with respect to price as a constant. Then it was interpreted as when the cost of production increases, the demand for both factors will decrease proportionally. But the writer neither named the function form constant elasticity of substitution nor emphasize the merit of this result.

It was only two decades later in \cite{Solow1956} that the CES made another significant appearance in literature. Author used CES production function as one possible example of production function to illustrate what later became one of the most central models of economic growth in modern growth theory.

Nonetheless, it wasn't until \cite{ACMS1961} article on substitution of production factors that the CES became widely used. \cite{ACMS1961} used an intercountry CES production function to show an over-all elasticity of substitution between capital and labor are significantly less than unity of all non-farm production in the U.S. They compared the three-digit industries in several countries (United States, Canada, United Kingdoms, Japan and India) and and compared the two-digit industries between Japan and U.S.. This model coincides with an wide and exponential popularization of CES functions in literature (See Figure) with short term increase in interest by the CES coming both from theoretical developments as well as applications of the function.

%%%%%%%%%%%%%%%%%%%%%%%%%%%%%%%%%%%%%%%%%%%%%%%%%%%%%%%%%%%%%%%%%%%%%%